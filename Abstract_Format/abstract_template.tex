% This is a LaTeX template 
% Last update: 22 Oct 2014

\documentclass[a4paper,11pt]{article}

%%%%%%% PACKAGES (PLEASE INSERT YOUR OWN TITLE, AUTHORS, AND KEYWORDS) %%%%%%%

\usepackage{multicol}
\usepackage{graphicx}
\usepackage{setspace}
\usepackage[
   pdftitle={CMS at LHC.}, %%% <- your title
   pdfauthor={Author One, Author Two, and so on},         %%% <- your author list
   pdfkeywords={photons, electrons, atoms, collisions},   %%% <- your keywords
   colorlinks,
   linkcolor=blue,
   urlcolor=blue,
   citecolor=blue,
   a4paper=true,
%   dvipdfm           % Uncomment if using dvipdfm to convert dvi into pdf
   ]{hyperref}

%%%\usepackage[para]{footmisc} %%% <- uncomment if you want several e-mail addresses to be inlined in the footer

%%%%%%%%%%%%%%%%%%%% DIMENSIONS (do not change) %%%%%%%%%%%%%%%%%%%%

\voffset=0.0cm
\hoffset=0.0cm

\topskip=0cm
\topmargin=-0.54cm
\oddsidemargin=-0.54cm
\evensidemargin=2.0cm

\textwidth=17.0cm
\textheight=25.7cm
\headheight=0cm
\headsep=0cm
\footskip=0cm

%%%%%%%%%%%%%%%%%%%% MACROS (do not change) %%%%%%%%%%%%%%%%%%%%

\renewcommand\refname{\normalsize References}
\renewcommand\baselinestretch{1}
\pagestyle{empty}
\newcommand{\abstracttitle}[1]{
 \begin{center}{\Large {\bf #1}}\end{center}
}
\newcommand{\authors}[1]{
 \vspace*{-0.3cm}
 \begin{center} {\bf #1} \end{center}
 \vspace*{-0.3cm}
}
\newcommand{\addresses}[1]{
 \begin{center} {\small #1} \end{center}
}
\newcommand{\synopsis}[1]{
 \begin{center}
 \setstretch{0.75}
 \begin{minipage}[t]{16cm}
   {\footnotesize {\bf Synopsis} #1 }
 \end{minipage}
 \setstretch{1.0}
 \end{center}
}
\newcommand{\abstracttext}[1]{
 \vspace*{-0.3cm}
 \columnsep0.75cm
 \begin{multicols}{2} #1 \end{multicols}
}
% Use \picturelportrait{0} when you want to include a portrait figure!
\newcommand{\pictureportrait}[2]{
 \vspace*{0.5cm}
 \centerline{
  \includegraphics*[width=7.8cm,angle=#1]{#2}
%%%  \includegraphics*[width=7.8cm,height=9cm,angle=#1]{#2}
 }
}
% Use \picturelandscape{0} when you want to include a landscape figure!
\newcommand{\picturelandscape}[2]{
 \vspace*{0.5cm}
 \centerline{
  \includegraphics*[width=7.8cm,angle=#1]{#2}
%%%  \includegraphics*[width=7.8cm,height=5.5cm,angle=#1]{#2}
 }
}
\newcommand{\capt}[2]{
 \vspace*{-0.3cm}
 \begin{center}
 \begin{minipage}[t]{7.8cm} {\small {\bf Figure~#1}.~#2} \end{minipage}
 \end{center}
 \vspace*{0.3cm}
}

\newcommand{\captTab}[2]{
 \vspace*{-0.1cm}
 \begin{center}
 \begin{minipage}[t]{7.8cm} {\small {\bf Table~#1}.~ #2} \end{minipage}
 \end{center}
 \vspace*{0.1cm}
}

\newcommand{\writeto}[1]{
 \hspace*{-2.5mm} \footnote{E-mail: \href{mailto:#1}{#1}}\hspace*{-1.5mm}
}

%%%%%%%%%%%%%%%%% USER'S INPUT IS EXPECTED BELOW THIS LINE %%%%%%%%%%%%%%%%%

\begin{document}




\abstracttitle{
CMS at LHC
}

\authors{
Diego Baron$^{\ast}$\writeto{diego.baron@udea.edu.co}


}

\addresses{
$^\ast$ Instituto de F\'{\i}sica, Universidad de Antioquia, Medell\'{\i}n, Colombia \\
}

\synopsis{
For experimental particle physicists is crucial to understand how an experiment like LHC is carried out, in order to know how and where to search for new physics. This seminar is focused in explaining how the LHC and in particular the CMS experiment works, we are goint to describe the process of collision, from the acceleration chain of protons to what kind of detectors are mounted in order to reconstruct the collisions and trayectories of particles. We are also going to describe the limitations of the CMS detector and how radiation damage involve constant recalibration of the systems.
}



\abstracttext{ After the Superconducting Super Collider was canceled, the 16 of December of 1994 the Large Hadron Collider (LHC) was approved and thus turned into the biggest and most important particle physics experiment. LHC uses the same 27 km tunnel that was caved for LEP (Large Electron-Positron Collider) , the LHC depth ranges from 50 to 175 m and is capable of accelerating protons and lead ions to an energy of 7 TeV per beam.
	
Collisions in this particle accelerator occur in four points, at each one, is located one of the four principal experiments conducted by CERN: ATLAS,CMS,ALICE and LHCb. We are going to put our efforts into explaining the Compact Muon Solenoid (CMS) experiment, it is the second biggest experiment on LHC and currently Universidad de Antioquia is part of the collaboration with te participation of 3 persons.

CMS is a detector located at point 5 of LHC ring, the detector has a diameter of 15m and a longitude of 29m, it weights 14000 Tons. Its called "Compact Solenoid" because some of the detections systems are inside of his 367.5 $m^3$ solenoid capable of producing a 3.8 T magnetic field and "Muon" due to its systems designed to reconstruct muon trayectories with very high precision [1,2,3].

The main subsystems of the CMS detector are: the tracker system, the superconducting solenoid, the electromagnetic calorimeter (ECAL), the hadronic calorimeter (HCAL) and the muon chambers (a 3-D representation is shown in Figure 1). All these systems produce an extraordinary amount of data (1 PB/s) and not all events are interesting, so an "inteligent" filter must be applied, the CMS detector has a system called the Trigger system that does this job in two stages and achieve a huge reduction of the data, by making the already mentioned selection [4].

The motivation of this talk is to present the LHC and CMS functioning, this is important because when an analysis of some particle physics model is going to be done, particle physicist have to know well what they can measure and with how many precission. As an really simple example, neutrinos escape from the detector, so you cannot look for his trayectories, another important task where it must be known very well how CMS operates, is making an accurate simulation of the detector, this have to be done every time before perform a realistic search in the detector.








%\picturelandscape{0}{ToledoTempJuly.eps}
\picturelandscape{0}{cms3d} % Use this for pdflatex
\capt{1}{CMS detector 3-D representation showing each of its subsystems.}



\begingroup
\small
\begin{thebibliography}{9}

\bibitem{diaz09} Jose David Ruiz,2015, Search for a vector-like quark T decaying into top+Higgs in single production mode in full hadronic final state using CMS data collected at 8 TeV.
\bibitem{diaz09} Tanja Rommerskirchen,2006, Study of Muonic Decays of Doubly Charged Higss Bosons with the CMS Detector.
\bibitem{diaz09} Werner Herr and Bruno Muratori,Concept of luminosity.
\bibitem{diaz09} https://cms.cern/.
\end{thebibliography}
\endgroup
}
  % end of the body

%\fi
\end{document}
